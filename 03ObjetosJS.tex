% Prof. Dr. Ausberto S. Castro Vera
% UENF - CCT - LCMAT - Curso de Ci\^{e}ncia da Computa\c{c}\~{a}o
% Campos, RJ,  2021
% Disciplina: Paradigmas de Linguagens de Programa\c{c}\~{a}o


\chapter{ Programa\c{c}\~{a}o Orientada a Objetos com JavaScript}


	\section{Módulos}
	Para tornar o código extensível, reutilizável e acessível, é interessante organizá-lo em classes. Porém, no JavaScript as classes não são o único tipo de código modular. Geralmente, um módulo é um único arquivo de JavaScript, e qualquer pedaço escrito na linguagem pode ser um módulo.
	Para acessar um módulo primeiro temos que exportá-lo, e isso é feito com a palavra "export". Abaixo, temos um exemplo de um método. \newline \newline
	
	\begin{lstlisting}
	export const nome = 'triangulo'
	
	export function desenha(forma ,tamanho, x, y, cor)
		forma.fillStyle = cor;
		forma.fillTriangulo(x, y, tamanho)
		
		return {
			tamanho: tamanho,
			x: x,
			y: y,
			cor:cor
		};
	}
	\end{lstlisting}
   %%%%%%%%======================
    \section{Classes e Objetos}
    %%%%%%%%======================
De acordo com \cite{flanagan2020javascript}, objetos são o tipo de dados fundamentais do JavaScript. Qualquer valor que não seja um tipo "true", "false", "null" ou "undefined", é um objeto. Isso nada mais é do que um valor composto que é constituído de múltiplos valores. Sendo assim, um objeto permite seu armazenamento e sua busca pelo nome. Na linguagem, os objetos são dinâmicos, o que significa que propriedades podem ser adicionadas ou removidas. 


\subsection{Listas}
Listas são um conjunto de dados e características armazenados dentro de uma variável. Os conteúdos de uma lista podem ser acessados através do index dos elementos. Abaixo estão alguns exemplos de como as linstas funcionam no JavaScript: \newline \newline

\begin{lstlisting}
//Cria uma lista com esses elementos
let Alimentos = ['Banana', 'Laranja', 'Melancia', 'Mexirica']  
//imprime o segundo elemento da lista de alimentos (Laranja) 
console.log(Alimentos[1])  
\end{lstlisting}

Ambos os objetos e as listas são tipos de dados que podem ser alterados e utilizados para armazenar vários valores. Os objetos servem para representar algo que pode ser definido ujunto à suas características. Por exemplo: um ser humano pode ter seu nomee idade e, além disso, seus comportamentos herdados de seus pais. A abstração do que é um ser humano é feita utilizando uma classe, que funcionará como um molde para o objeto criado. \newline
Diferente de um objeto, uma lista serve apenas como um meio de armazenar dados em uma única variável. Nesse sentido, os conceitos da orientação a objeto, como herança e polimorfismo, não seriam possíveis de serem implementados dentro de uma lista.

\subsection{Propriedades}
	Uma propriedade tem nome e valor, e seu nome pode ser uma string porém não pode existir um objeto que tenha mais de uma propriedade com mesmo nome. Os valores das propriedades podem ser quaisquer que existam dentro da linguagem. \newline
	Além de nome e valor, cada propriedade tem valores associados que chamados de atributos de propriedades. 
	
	\subsubsection{Atributos}
	Segundo \cite{flanagan2020javascript}, o JavaScript apresenta os seguintes atributos de propriedades:
	"writable" diz se o valor da propriedade pode ser atribuido. Caso seja falso, o valor da propriedade não pode ser alterado.
	"enumerable" diz se o nome da propriedade é retornado por um for/in loop. Se verdadeiro, a propriedade aparece durante a enumeração das propriedades do objeto correspondente. 
	"configurable" especifica se a propriedade pode ser deletada ou se seus atributos podem ser alterados.
\subsection{Criando Objetos}
	A linguagem JavaScript apresenta várias formas de criar um objeto, e uma forma simples e fácil de criá-los é inserindo um literal de objeto. Essa é uma forma extremamente prática e intuitiva, o que torna a programação orientada a objetos na linguagem muito mais simples. \newline

	Um literal de objeto contém a propriedad e o seu valor, seguido de vírgulas. Abaixo, um exemplo de um literal de objeto: \newline \newline
   \begin{lstlisting}
    var objeto = {
    	primeiraPropriedade: "Caracteristica 1",
    	segundaPropriedade: 101,
    	terceiraPropriedade: false,
    	data: {
    		dia: 12,
    		ano: 2003
    	}
    
    }
    \end{lstlisting}

	Além disso, há também o operador "new", que cria e inicializa um objeto. Para isso, a palavra reservada "new" vem seguida de uma chamada de função. Essa função é chamada de função construtora e tem como objetivo a inicialização do novo objeto.
	
	\begin{lstlisting}
		var objeto = new Object() //Cria um objeto vazio {}
	\end{lstlisting}
	
	\subsection{Lendo e adicionando propriedades}
	Para obter valores de objetos utilizamos o ponto (.) ou colchetes ([]). O exemplo abaixo adiciona propriedades a um objeto chamado pessoa, lê e as coloca em variáveis.
	\newline
	\newline
	\begin{lstlisting}
	
	var pessoa = new Object()  //Cria um objeto pessoa vazio
	pessoa.nome = "Marcelo"
	pessoa.idade = 8
	pessoa.sexo = "M"
	
	console.log(pessoa.nome)	//Mostra o valor da propriedade nome de pessoa
	console.log(pessoa.idade)	//Mostra o valor da propriedade idade de pessoa
	console.log(pessoa.sexo) 	//Mostra o valor da propriedade sexo de pessoa

	console.log(pessoa["nome"]) //E o mesmo que o codigo da linha 7
	\end{lstlisting}
	\subsection{Deletando Propriedades}
	O operador "delete" remove uma propriedade de um objeto. Isso significa que se um objeto tem uma propriedade, o seu conteúdo não será deletado, mas sim a propriedade em si. O exemplo abaixo ilustra o que aconteceria ao apagar uma propriedade de um objeto existente: \newline
	\newline
	\begin{lstlisting}
	//Cria um novo objeto chamado pessoa
	var pessoa = new Object()
	
	//define a propriedade "nome" como sendo "Lucas"
	pessoa.nome = "Lucas"
	//define a propriedade "idade" valendo 18
	pessoa.idade = 18
	//define  a profissao
	pessoa.profissao = "Engenheiro"
	//define o cpf
	pessoa.cpf = "123.456.789-00"
	
	//printa a propriedade cpf do objeto pessoa
	console.log(pessoa.cpf) 
	//deleta a propriedade cpf do objeto pessoa
	delete pessoa.cpf
	//printa a propriedade "cpf" de pessoa, porem, como essa propriedade foi deletada, o console ira retornar "undefined"
	console.log(pessoa.cpf) 
	\end{lstlisting}
   %%%%%%%%======================
	
	\subsection{Prototype}
	Como abordado por \cite{flanagan2020javascript}, uma classe é um conjunto de objetos que herdam propriedades do mesmo objeto prototype. O objeto prototype é herdado por todo objeto criado, e todas as classes herdam dele.
	
   %%%%%%%%======================
    \section{Heran\c{c}a}
    %%%%%%%%======================
	Um dos conceitos mais importantes da programação orientada a objetos é a Herança. Ela serve para que um objeto consiga herdar características de um objeto mãe. Isso permite que o código não necessite de ser reescrito. Na linguagem, cada objeto tem um conjunto de propriedades próprias, e elas também herdam propriedades de seu objeto prototype. No exemplo abaixo, temos o exemplo de uma classe "Carro" que herda da classe "Veiculo": \newline
	\newline
	\begin{lstlisting}
	class Carro extends Veiculo {
		rodas = 4;
		cor = "Vermelho";
	}
	\end{lstlisting}
	
	\section{Encapsulamento}	
	Por definição, o encapsulamento é o processo de esconder dados. Isso acontece porque nem sempre é seguro ou interessante permitir que determinados dados sejam acessados por qualquer um dentro do programa, e por isso costumamos separar a implementação através de uma interface. Basicamente, o processo de encapsulamento traz uma camada de segurança e confiabilidade ao código. No JavaScript é permitido utilizar variáveis privadas para permitir o encapsulamento. A linguagem permite a utilização de getters e setters que não podem ser deletados.

	
   %%%%%%%%======================
    
    %%%%%%%%====================== 
% Prof. Dr. Ausberto S. Castro Vera
% UENF - CCT - LCMAT - Curso de Ci\^{e}ncia da Computa\c{c}\~{a}o
% Campos, RJ,  2021
% Disciplina: Paradigmas de Linguagens de Programa\c{c}\~{a}o


\chapterimage{Conclusao.jpg} % Chapter heading image
\chapter{Conclus\~{o}es}


Fazer este trabalho ao longo do semestre com certeza não foi uma tarefa fácil. Foram longas noites pensando no que seria escrito, aprendendo a usar o LaTeX e revisando o conteúdo produzido. A pesquisa e elaboração do trabalho foram árduas e, mesmo com certo grau de experiência na linguagem, encontrei muitos conceitos que até então eram desconhecidos. 
Os livros da linguagem com certeza ajudaram muito, porém nem sempre foram a única fonte de referências utilizada. Como diria Richard Feynman, se você não consegue explicar algo em termos simples, você não entende do assunto. Logo, para explicar algo, primeiro é preciso entender. Nesse sentido, para reforçar meu conhecimento foram assistidos tutoriais em video da linguagem, além da leitura da documentação oficial da linguagem.
Ademais, aprender esses conceitos foi a parte mais difícil do trabalho.
Contudo, foi a busca incessante de referências e conhecimentos que fizeram a experiência de desenvolver esse trabalho ser tão gratificante. Por isso, explicar algo é sempre a melhor forma de aprender, e sendo assim, escrever este livro foi uma excelente forma de aprendizado.  
  
  O trabalho foi desenvolvido com o objetivo de explicar e introduzir a linguagem de forma simples e resumida. Portanto, não foi possível explicar detalhadamente alguns conceitos como os pontos flutuantes binários e os erros de arredondamento. 



   \begin{figure}[H]
    \begin{center}
        \caption{Linguagens de programa\c{c}\~{a}o modernas} \label{ling2}
        \includegraphics[width=12cm]{js3.png} \\
        {\tiny \sf Fonte: O autor }
    \end{center}
   \end{figure} 
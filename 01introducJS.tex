% Prof. Dr. Ausberto S. Castro Vera
% UENF - CCT - LCMAT - Curso de Ci\^{e}ncia da Computa\c{c}\~{a}o
% Campos, RJ,  2021
% Disciplina: Paradigmas de Linguagens de Programa\c{c}\~{a}o




\chapter{ Introdu\c{c}\~{a}o}

A linguagem de programação JavaScript é a “linguagem da web”. Seu uso é dominante na internet e praticamente quase todos os sites a utilizam. 
Além disso, smartphones, tablets e vários outros dispositivos têm interpretadores de JavaScript embutidos. 
Isso a torna uma das linguagens mais utilizadas dos dias atuais e uma das linguagens mais usadas por desenvolvedores de software. 
É importante dizer que, embora o nome sugira, JavaScript é uma linguagem completamente diferente e independente da linguagem Java. 
Mesmo assim, suas sintaxes tem traços de semelhança, mas nada além disso. \newline

Por ser uma linguagem fácil de ser aprendida e fortemente tolerante, permitiu que usuários pudessem ter suas necessidades 
atendidas de forma cômoda e eficiente.
 A linguagem é de alto-nível, dinâmica e interpretada. Além disso, é adequada para orientação a objeto e programação funcional. 
 É uma linguagem não tipada – ou seja, suas variáveis não tem um tipo específico e seus tipos não são importantes para a linguagem. 
Baseado no livro \cite{flanagan2020javascript}.

\section{Aspectos hist\'{o}ricos da linguagem JavaScript}
   
A linguagem foi criada na NETSCAPE por Brendan Eich. Tecnicamente, JavaScript é uma marca registrada da Sun Microsystems (atualmente Oracle) usada para descrever a implementação da língua pela Netscape (atualmente Mozilla). Na época, a Netscape enviou a linguagem para a padronização da ECMA – European Computer Manufacturer’s Association, e sua versão padronizada ficou conhecida como “ECMAScript”. Na prática, todos chamam a linguagem apenas de JavaScript.
De acordo com \cite{flanagan2020javascript}.



   \section{\'{A}reas de Aplica\c{c}\~{a}o da Linguagem}

   A linguagem JavaScript é completamente versátil e tem aplicações nos mais variados ambientes, seja no client-side ou no server-side. 
   Nesta seção
   falarei de algumas aplicações e paradigmas da programação que podem ser implementados em JavaScript.

        \subsection{NodeJS}
        A linguagem foi criada para ser utilizada em navegadores da web, e esse segue sendo seu ambiente mais comum de execução até hoje. Enfim, o ambiente do navegador permite a linguagem obter a entrada de usuários e fazer requests HTTP. Porém, em 2010 outro ambiente foi criado para executar código em JavaScript. 
O NodeJS, popularmente conhecido como Node, tinha a ideia de invés de manter a linguagem presa a um navegador, 
permitir que a linguagem tivesse acesso ao sistema operacional. Isso proporcionou a utilização da linguagem no lado do servidor, invés de se limitar apenas ao navegador. Atualmente, o Node tem grande popularidade na implementação de servidores web.
Baseado no livro \cite{flanagan2020javascript}.
        \subsection{ Orienta\c{c}\~{a}o a objetos}
        A linguagem é orientada a objeto, porém apresenta algumas diferenças que valem ser mencionadas. 
        Na linguagem, as classes são baseadas no mecanismo de herança de protótipos. Se dois objetos herdam do mesmo objeto protótipo, então diz-se que são instâncias de uma mesma classe. 
        Membros, ou instâncias da classe, tem suas propriedades para manter e também métodos que definem seu comportamento. Este comportamento é definido pela classe e compartilhado para todas as instancias.
Retirado do artigo da documentação da linguagem, em: \url{https://developer.mozilla.org/en-US/docs/Web/JavaScript/Reference/Classes}
         \subsection{Programação Funcional}
         Basicamente, a programação funcional é um paradigma da programação que visa produzir software através de funções puras, 
         evitando compartilhamento de estados, dados mutáveis e efeitos colaterais.
         Embora JavaScript não seja uma lingugagem de programação funcional como Haskel ou Lisp, 
         o fato da linguagem poder manipular funções como objeto significa que técnicas de programação funcional podem ser implementadas na linguagem.
         Os metodos de array do ECMAScript 5, como map() e reduce() satisfazem bem o estilo de programação funcional. 
         Retirado do livro \cite{powers2015javascript}.
       
% Prof. Dr. Ausberto S. Castro Vera
% UENF - CCT - LCMAT - Curso de Ci\^{e}ncia da Computa\c{c}\~{a}o
% Campos, RJ,  2021
% Disciplina: Paradigmas de Linguagens de Programa\c{c}\~{a}o


\chapter{Ferramentas existentes e utilizadas}

Neste cap\'{\i}tulo devem ser apresentadas pelo menos DUAS (e no m\'{a}ximo 5) ferramentas consultadas e utilizadas para realizar o trabalho, e usar nas aplica\c{c}\~{o}es. Considere em cada caso:
\begin{itemize}
  \item Nome da ferramenta (compilador-interpretador)
  \item Endere\c{c}o na Internet
  \item Vers\~{a}o atual e utilizada
  \item Descri\c{c}\~{a}o simples (m\'{a}x 2 par\'{a}grafos)
  \item Telas capturadas da ferramenta
  \item Outras informa\c{c}\~{o}es
\end{itemize}

    \section{Node JS}
    A linguagem JavaScript não está mais atrelada somente ao navegador. Por isso, para utilizar a linguagem sem precisar recorrer a um Browser, utiliza-se o nodeJS. O NodeJS, conhecido apenas como Node, nada mais é do que o V8 (Engine do JavaScript no navegador Google Chrome) fora do Chrome. Sendo assim, para instalar o Node basta entrar em https://nodejs.org e baixar a versão desejada. 
    
\begin{figure}[h]
	\centering
	\includegraphics[width=0.3\linewidth]{Pictures/NodeLogo}
	\caption{}
	\label{fig:nodelogo}
\end{figure}
    
    \subsection{NVM}
    Devido a existência de várias versões do Node, o NVM - Node Version Manager - facilita o trabalho com versões diferentes. Por exemplo, se num projeto antigo decide-se pela versão 11.5, é possível com o NVM utilizar o node 11.5 sempre naquele projeto, mesmo após atualizar o node para versões mais recentes.


    \section{Visual Studio Code}
    Hoje em dia existem diversas versões de IDEs para as mais variadas linguagens de programação. Netbeans e Eclipse para o Java, Pycharm para o Python entre outras. IDEs feitas pensando especificamente em uma linguagem sempre tiveram vantagem em relação às IDEs multi-uso, como o sublime-text. Porém, com o Visual Studio Code, que chamarei de VSC, programar em várias linguagens é fácil e sem dor de cabeça. 
    O VSC é uma IDE fácil de usar, relativamente leve e permite a instalação de inúmeras extensões para tornar o desenvolvimento o mais produtivo possível.
    Portanto, para programar em JavaScript, o VSCode é uma das melhores IDEs disponíveis. É fácil programar para web utilizando JS, HTML e CSS, e além disso, vários recursos são disponibilizados pelo programa.
    
	\begin{figure}[h]
		\centering
		\includegraphics[width=0.2\linewidth]{Pictures/VSC_Logo}
		\caption{}
		\label{fig:vsclogo}
	\end{figure}


    \section{Interpretador UVW}


    \section{Ambientes de Programa\c{c}\~{a}o IDE MNP} 